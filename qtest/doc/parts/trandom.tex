
Quickcheck is a small library useful for randomized unit tests. Using it is a bit more
complex, but much more rewarding than simple tests.

\begin{verbatim}
(*$Q <header>
  <generator> (fun <generated value> -> <statement>)
  ...
*)
\end{verbatim}

Let us dive into an example straight-away:

\begin{verbatim}
(*$Q foo
  Q.small_int (fun i-> foo i (+) [1;2;3] = List.fold_left (+) i [1;2;3])
*)
\end{verbatim}

The Quickcheck module is accessible simply as \emph{Q} within inline tests; \texttt{small\_int} is a
generator, yielding a random, small integer. When the test is run, each statement will be
evaluated for a large number of random values -- $100$ by default. Running this test for the
above definition of foo catches the mistake easily:

{\Oconsole\begin{verbatim}
law foo.ml:14::>  Q.small_int (fun i-> foo i (+) [1;2;3]
    = List.fold_left (+) i [1;2;3])
failed for 2
\end{verbatim} }

Note that the random value for which the test failed is provided by the error message --
here it is 2. It is also possible to generate several random values simultaneously using
tuples. For instance


\begin{verbatim}
(Q.pair Q.small_int (Q.list Q.small_int)) \
  (fun (i,l)-> foo i (+) l = List.fold_left (+) i l)
\end{verbatim}

will generate both an integer and a list of small integers randomly. A failure will then
look like

{\Oconsole\begin{verbatim}
law foo.ml:15::>  (Q.pair Q.small_int (Q.list Q.small_int))
    (fun (i,l)-> foo i (+) l = List.fold_left (+) i l)
failed for (727, [4; 3; 6; 1; 788; 49])
\end{verbatim}}

\textbf{Available Generators:}

\begin{itemize}
\item  \underline{Simple generators:}\\
\texttt{unit}, \texttt{bool}, \texttt{float}, \texttt{pos\_float},
\texttt{neg\_float}, \texttt{int}, \texttt{int32}, \texttt{int64}, \texttt{pos\_int}, \texttt{small\_int},
\texttt{neg\_int}, \texttt{char}, \texttt{printable\_char}, \texttt{numeral\_char},
\texttt{string}, \texttt{printable\_string},
\texttt{numeral\_string}

\item \underline{Structure generators:}\\
\texttt{list} and \texttt{array}. They take one generator as their argument. For instance
\texttt{(Q.list Q.neg\_int)} is a generator of lists of (uniformly taken) negative integers.

\item \underline{Tuples generators:}\\
\texttt{pair} and \texttt{triple} are respectively binary and ternary. See above for an example of
\texttt{pair}.

\item \underline{Size-directed generators:}\\
\texttt{string}, \texttt{numeral\_string}, \texttt{printable\_string}, \texttt{list}
and \texttt{array} all have \texttt{*\_of\_size}
variants that take the size of the structure as their first argument.
\end{itemize}

\textbf{Tips:}

\begin{itemize}
\item \textbf{Duplicate Elements in Lists:}
When generating lists, avoid \texttt{Q.list Q.int} unless you have a good reason to do so.
The reason is that, given the size of the \texttt{Q.int} space, you are unlikely
to generate any duplicate elements. If you wish to test your function's behaviour with
duplicates, prefer \texttt{Q.list Q.small\_int}.

\item \textbf{Changing Number of Tests:}
If you want a specific test to execute each of its statements a specific number of times
(deviating from the default of $100$), you can specify it explicitly through
\emph{parameter injection} (cf. Section \ref{sec:param}) using the
\texttt{count} : \textsf{int} argument.

\item \textbf{Getting a Better Counterexample:}
By default, a random test stops as soon as one of its generated values
yields a failure. This first failure value is probably not the best possible
counterexample. You can \emph{force} qtest to generate and test all \texttt{count}
random values regardless, and to display the value which is smallest with respect
to a certain measure which you define. To this end, it suffices to use parameter injection
to pass argument \texttt{small} : $\alpha \to \beta$,
where $\alpha$ is the type of
generated values and $\beta$ is any totally ordered set (wrt. $<$). Typically
you will take $\beta = \textsf{int}$ or $\beta = \textsf{float}$. Example:
\begin{verbatim}
let fuz x = x
let rec flu = function
  | [] -> []
  | x :: l -> if List.mem x l then flu l else x :: flu l

(*$Q fuz; flu & ~small:List.length
  (Q.list Q.small_int) (fun x -> fuz x = flu x)
*)
\end{verbatim}
The meaning of \texttt{~small:List.length} is therefore simply:
``choose the shortest list''. For very complicated cases, you can simultaneously
increase \texttt{count} to yield an even higher-quality counterexample.

\end{itemize}


